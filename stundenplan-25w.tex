\documentclass{stundenplan}

\spSetStartTime{8}  % Früheste Uhrzeit
\spSetEndTime{19}   % Späteste Uhrzeit
\spSetHeadline{Wintersemester 2025/26}

\begin{document}

\begin{stundenplan}

% Software Engineering als Typ A (blau/hellblau)
\eventA{Software-Technik \&\\ Software Engineering}{S 01.213}{Mo}{14:00}{15:30}
\eventA{Software-Technik \&\\ Software Engineering}{S 07.210}{Di}{15:45}{17:15}
\eventAalt{Labor Software-Technik \&\\ Software Engineering}{S 07.210}{Di}{17:30}{19:00}

% Informatik 2 als Typ B (grün/hellgrün)
\eventB{Informatik 2}{S 01.007}{Mo}{15:45}{17:15}
\eventB{Informatik 2}{S 01.006}{Di}{11:30}{13:00}
\eventBalt{Labor Informatik 2}{S 07.001}{Mo}{9:45}{11:15}

% ASM als Typ D (rot) - könnte auch Typ C oder Generic sein
\eventD{ASM Software Engineering}{S 07.101}{Do}{14:00}{15:30}
\eventD{ASM Software Engineering}{S 07.101}{Do}{15:45}{17:15}

% Meetings als generische Events (grau)
\eventGeneric{Dekanats JF}{}{Mi}{12:00}{13:00}
\eventGeneric{Fakultätsrat}{}{Mi}{14:30}{17:00}
\eventGeneric{Industriekolloquium}{}{Mi}{17:00}{18:00}

\end{stundenplan}

\end{document}